%%%%%%%%%%%%%%%%%%%%%%%%%%%%%%%%%%%%%%%%%
% Beamer Presentation
% LaTeX Template
% Version 1.0 (10/11/12)
%
% This template has been downloaded from:
% http://www.LaTeXTemplates.com
%
% License:
% CC BY-NC-SA 3.0 (http://creativecommons.org/licenses/by-nc-sa/3.0/)
%
%%%%%%%%%%%%%%%%%%%%%%%%%%%%%%%%%%%%%%%%%

%----------------------------------------------------------------------------------------
%	PACKAGES AND THEMES
%----------------------------------------------------------------------------------------

\documentclass{beamer}

\mode<presentation> {

% The Beamer class comes with a number of default slide themes
% which change the colors and layouts of slides. Below this is a list
% of all the themes, uncomment each in turn to see what they look like.

%\usetheme{default}
%\usetheme{AnnArbor}
%\usetheme{Antibes}
%\usetheme{Bergen}
%\usetheme{Berkeley}
%\usetheme{Berlin}
%\usetheme{Boadilla}
%\usetheme{CambridgeUS}
%\usetheme{Copenhagen}
%\usetheme{Darmstadt}
%\usetheme{Dresden}
%\usetheme{Frankfurt}
%\usetheme{Goettingen}
%\usetheme{Hannover}
%\usetheme{Ilmenau}
%\usetheme{JuanLesPins}
%\usetheme{Luebeck}
\usetheme{Madrid}
%\usetheme{Malmoe}
%\usetheme{Marburg}
%\usetheme{Montpellier}
%\usetheme{PaloAlto}
%\usetheme{Pittsburgh}
%\usetheme{Rochester}
%\usetheme{Singapore}
%\usetheme{Szeged}
%\usetheme{Warsaw}

% As well as themes, the Beamer class has a number of color themes
% for any slide theme. Uncomment each of these in turn to see how it
% changes the colors of your current slide theme.

%\usecolortheme{albatross}
%\usecolortheme{beaver}
%\usecolortheme{beetle}
%\usecolortheme{crane}
%\usecolortheme{dolphin}
%\usecolortheme{dove}
%\usecolortheme{fly}
%\usecolortheme{lily}
%\usecolortheme{orchid}
%\usecolortheme{rose}
%\usecolortheme{seagull}
%\usecolortheme{seahorse}
%\usecolortheme{whale}
%\usecolortheme{wolverine}

%\setbeamertemplate{footline} % To remove the footer line in all slides uncomment this line
%\setbeamertemplate{footline}[page number] % To replace the footer line in all slides with a simple slide count uncomment this line

%\setbeamertemplate{navigation symbols}{} % To remove the navigation symbols from the bottom of all slides uncomment this line
}

\usepackage{graphicx} % Allows including images
\usepackage{booktabs} % Allows the use of \toprule, \midrule and \bottomrule in tables
\usepackage{subfigure}
\usepackage{amsmath}
\usepackage{mathtools}
%----------------------------------------------------------------------------------------
%	TITLE PAGE
%----------------------------------------------------------------------------------------

\title[multithreaded test synthesis]{Multithreaded Test Synthesis for Deadlock Detection} % The short title appears at the bottom of every slide, the full title is only on the title page

\author{Alexander Zhdanov} % Your name
\institute[TUD.SOLA] % Your institution as it will appear on the bottom of every slide, may be shorthand to save space
{Technical University Darmstadt\\ Software Lab \\ % Your institution for the title page
\medskip
\textit{azhdanov@hotmail.com} % Your email address
}
\date{\today} % Date, can be changed to a custom date

\begin{document}

\begin{frame}
\titlepage % Print the title page as the first slide
\end{frame}

\begin{frame}
\frametitle{Overview} % Table of contents slide, comment this block out to remove it
\tableofcontents % Throughout your presentation, if you choose to use \section{} and \subsection{} commands, these will automatically be printed on this slide as an overview of your presentation
\end{frame}

%----------------------------------------------------------------------------------------
%	PRESENTATION SLIDES
%----------------------------------------------------------------------------------------

%------------------------------------------------
\section{Introduction} % Sections can be created in order to organize your presentation into discrete blocks, all sections and subsections are automatically printed in the table of contents as an overview of the talk
%------------------------------------------------

\begin{frame}
\frametitle{Illustrative example}
\begin{figure}
\includegraphics[scale = 0.5]{Images/Illustrative_example}\\
\caption{Illustrative example}
\end{figure}
Implementation of A is not thread-safe: testFoo(a1, a2) and testFoo(a2, a1) are invoked concurrently
\end{frame}

\begin{frame}
\frametitle{Theirs contribution}

    \begin{itemize}
    \item automatic multithreaded test synthesis for deadlock detection
    \item identification of method call queues and their invocation context
    \item design and implemetation of OMEN
    \item evaluation on several multithreaded java libraries
    \end{itemize}

\end{frame}

%------------------------------------------------

%------------------------------------------------
\section{Motivation} % Sections can be created in order to organize your presentation into discrete blocks, all sections and subsections are automatically printed in the table of contents as an overview of the talk
%------------------------------------------------

%\begin{frame}
%\frametitle{Motivation}
%\begin{figure}[ht!]
%     \begin{center}
%
%        \subfigure[Implementation of sampleBootstrap]{%
%            \label{fig:first}
%           \includegraphics[scale = 0.3]{Images/sampleBootStrap}
%        }%
%        \subfigure[Sample usage of sampleBootstrap]{%
%           \label{fig:second}
%           \includegraphics[scale = 0.3]{Images/exampleSampleBootStrap}
%        }\\ %  ------- End of the first row ----------------------%
%
%    \end{center}
%    \caption{%
%        Motivation from the colt library
%     }%
%  \label{fig:subfigures}
%\end{figure}
%\end{frame}

\begin{frame}
\frametitle{Motivation}
\begin{figure}[ht!]
     \begin{center}
%
        \subfigure[Detected deadlocks]{%
           \label{fig:second}
           \includegraphics[scale = 0.3]{Images/colt_detected}
        }\\ %  ------- End of the first row ----------------------%
        \subfigure[Test case synthesized by OMEN]{%
            \label{fig:third}
           \includegraphics[scale = 0.3]{Images/testCaseOMEN}
        }%
%
    \end{center}
    \caption{%
        Motivation from the colt library
     }%
   \label{fig:subfigures}
\end{figure}

\end{frame}

\section{Design}
\begin{frame}
\frametitle{Design}
\begin{figure}[ht!]
     \begin{center}
%
       
      \includegraphics[scale = 0.4]{Images/design}
        
%
    \end{center}
    \caption{%
        Architecture of OMEN
     }%
   \label{fig:subfigures}
\end{figure}

\end{frame}

\subsection{Running example}
\begin{frame}
\frametitle{Running example}
\begin{figure}[ht!]
     \begin{center}
%
       
      \includegraphics[scale = 0.4]{Images/running_example}
        
%
    \end{center}
    \caption{%
        A running example
     }%
   \label{fig:subfigures}
\end{figure}
\end{frame}

\begin{frame}
\frametitle{Lock dependency relation}
The lock node is defined $\eta = (\tau, s, H, TI) $
\begin{itemize}
\item Lock type $(\tau)$: data type of the lock object
\item Source location $(s)$: source location of lock
\item Held locks $(H)$: set of objects that are currently held
\item Test and trace location identifier (TI): pair $(I_i, index)$, where $I_i$ is a testcase, $index$ is an index of the lock acquisition
\end{itemize}
\end{frame}

\subsection{Logger}
\begin{frame}
\frametitle{Logger}
\begin{figure}[ht!]
     \begin{center}
%
       
       \subfigure[Partial execution trace]{%
           \label{fig:second}
           \includegraphics[scale = 0.3]{Images/exec_trace}
        }
        \subfigure[D after executing $I_1$, $I_2$]{%
           \label{fig:second}
           \includegraphics[scale = 0.3]{Images/DI1I2}
        }
        \\ %  ------- End of the first row ----------------------%
        %\subfigure[Lock types]{%
        %    \label{fig:third}
        %   \includegraphics[scale = 0.3]{Images/lock_dependency}
        %}%
        
%
    \end{center}
    \caption{%
        Logger
     }%
   \label{fig:subfigures}
\end{figure}
\end{frame}

\subsection{Cycle detector}
\begin{frame}
\frametitle{Cycle detector}
\begin{figure}[ht!]
     \begin{center}
%
       
       \subfigure[Algorithm of cycle detector]{%
           \label{fig:second}
           \includegraphics[scale = 0.3]{Images/alg_cycle_det}
        }
        \subfigure[Paths and cycles generated for the running example]{%
           \label{fig:second}
           \includegraphics[scale = 0.3]{Images/paths_and_cycles}
        }
        \\ %  ------- End of the first row ----------------------%
        %\subfigure[Lock types]{%
        %    \label{fig:third}
        %   \includegraphics[scale = 0.3]{Images/lock_dependency}
        %}%
        
%
    \end{center}
    \caption{%
        Cycle detector
     }%
   \label{fig:subfigures}
\end{figure}
\end{frame}

\subsection{Synthesizer}
\begin{frame}
\frametitle{Synthesizer}
\begin{itemize}
 \item For each $eta \in Theta$, identify the associated method invocation in I
 \item Identify the parameters of m, on which the lock acquisitions are dependent
 \item Infer the constraints on the parameters of all  $m \in M$
\end{itemize}

\end{frame}

\begin{frame}
\frametitle{Synthesizer}
\begin{figure}[ht!]
     
           \includegraphics[scale = 0.3]{Images/mi_identifier}
       
    \caption{%
        Method invocation identifier
     }%
   \label{fig:subfigures}
\end{figure}
\end{frame}

\begin{frame}
\frametitle{Synthesizer}
\begin{figure}[ht!]
     
       \begin{center}
%
       
       \subfigure[Rules]{%
           \label{fig:second}
           \includegraphics[scale = 0.3]{Images/rules}
        }
        \subfigure[Parameter tracking for $eta3$ and $eta1$]{%
           \label{fig:second}
           \includegraphics[scale = 0.3]{Images/par_track}
        }
        \\ %  ------- End of the first row ----------------------%
        %\subfigure[Lock types]{%
        %    \label{fig:third}
        %   \includegraphics[scale = 0.3]{Images/lock_dependency}
        %}%
        
%
    \end{center}
       
    \caption{%
        Rules for identification of relevant parameters. $\oplus$ is a concatentation of an element to a sequence.
     }%
   \label{fig:subfigures}
\end{figure}
\end{frame}

\begin{frame}
\frametitle{Constraints generation}
 For a cycle $\theta = (\eta_1, \eta_2,..., \eta_k )$, where $\eta_i^f \to \eta_i$\\
 the following constraint set $C$ is generated for all $i \in [1,k]$:\\
 \begin{itemize}
  \item $P(\eta_{i-1}) \equiv P(\eta_i^f)$, if $i \in [2,k]$
  \item $P(\eta_k) \equiv P(\eta_i^f)$, otherwise
 \end{itemize}
 \begin{figure}[ht!]
     
  \includegraphics[scale = 0.3]{Images/cycle.png}
       
  \caption{%
        Intuitive explanation (cycle)
  }%
\end{figure}
\end{frame}

\subsection{Generator}
\begin{frame}
\frametitle{Generator}
\begin{figure}[ht!]
     
           \includegraphics[scale = 0.3]{Images/skel.png}
       
    \caption{%
        Skeleton of synthesized multithreaded test.
     }%
   \label{fig:subfigures}
\end{figure}
\end{frame}

\begin{frame}
\frametitle{Generator}
\begin{figure}[ht!]
     
       \begin{center}
%
       
       \subfigure[Enforce]{%
           \label{fig:second}
           \includegraphics[scale = 0.3]{Images/enforce}
        }
        \subfigure[Object assignments for the parameters]{%
           \label{fig:second}
           \includegraphics[scale = 0.3]{Images/oass}
        }
        \\ %  ------- End of the first row ----------------------%
        %\subfigure[Lock types]{%
        %    \label{fig:third}
        %   \includegraphics[scale = 0.3]{Images/lock_dependency}
        %}%
        
%
    \end{center}
       
    \caption{%
        Algorithm enforce.
     }%
   \label{fig:subfigures}
\end{figure}
\end{frame}

\section{Implementation}
\begin{frame}
\frametitle{Implementation}
Instrumentation:
 \begin{enumerate}
  \item $soot$ to generate execution traces
 \end{enumerate}
 
Cycle detection:
 \begin{enumerate}
  \item both types and subtypes
  \item each cycle can encode multiple deadlocks
  \item $iGoodLock$ for the deadlock detection
  \item random test case to identify a method invocation
 \end{enumerate}
 
 
\end{frame}

\section{Evaluation}
\begin{frame}
\frametitle{Experimental results}
\begin{figure}[ht!]
     
       \begin{center}
%
       
       \subfigure[Experimental results with $T_R$ as the seed testsuite. $D$: Lock dependency relation, $Sigma$: execution traces, $S$: cycle detection and parameter
       synthesis, $G$: generator, $DD$: deadlock detection, $Theta$: detected cycles, $CMI$: concurrent method invocations, $DL$: deadlocks, $TP$: true positives]{%
           \label{fig:second}
           \includegraphics[scale = 0.3]{Images/er1}
        }
        \subfigure[Benchmark Information. LoC: lines of code accross classes covered by test cases. $|M_total|$:number of public methods in class]{%
           \label{fig:second}
           \includegraphics[scale = 0.3]{Images/bench}
        }
        \\ %  ------- End of the first row ----------------------%
        %\subfigure[Lock types]{%
        %    \label{fig:third}
        %   \includegraphics[scale = 0.3]{Images/lock_dependency}
        %}%
        
%
    \end{center}
       
    \caption{%
        Experimental results
     }%
   \label{fig:subfigures}
\end{figure}
\end{frame}

\begin{frame}
\frametitle{Experimental results}
\begin{figure}[ht!]
    \includegraphics[scale = 0.3]{Images/er2}
    \caption{%
        Experimental results with $S$: cycle detection and parameter synthesis, $G$: generator, $DD$: deadlock detection, $Theta$: detected cycles,
        $CMI$: concurrent method invocations, $|T|$: tests generated, $DL$: deadlocks, $TP$: true positives
     }%
\end{figure}
\end{frame}

\section{Conclusions}
\begin{frame}
\frametitle{Conclusions}

    \begin{itemize}
    \item pros:
    \begin{enumerate}
      \item automatic
      \item effective
      \item ...
    \end{enumerate}
    \item cons:
    \begin{enumerate}
      \item false negatives (dynamic approach)
      \item failed to detect some deadlocks for an automatically generated test suite, on the other hand, manually written tests do not always have a good code coverage
      \item locks under consideration could be influenced by other method invocations
      \item does not handle aliasing and collections
    \end{enumerate}
    \end{itemize}

\end{frame}


%------------------------------------------------

%\begin{frame}
%\frametitle{Bullet Points}
%\begin{itemize}
%\item Lorem ipsum dolor sit amet, consectetur adipiscing elit
%\item Aliquam blandit faucibus nisi, sit amet dapibus enim tempus eu
%\item Nulla commodo, erat quis gravida posuere, elit lacus lobortis est, quis porttitor odio mauris at libero
%\item Nam cursus est eget velit posuere pellentesque
%\item Vestibulum faucibus velit a augue condimentum quis convallis nulla gravida
%\end{itemize}
%\end{frame}

%------------------------------------------------

%\begin{frame}
%\frametitle{Blocks of Highlighted Text}
%\begin{block}{Block 1}
%Lorem ipsum dolor sit amet, consectetur adipiscing elit. Integer lectus nisl, ultricies in feugiat rutrum, porttitor sit amet augue. Aliquam ut tortor mauris. Sed volutpat ante purus, quis accumsan dolor.
%\end{block}

%\begin{block}{Block 2}
%Pellentesque sed tellus purus. Class aptent taciti sociosqu ad litora torquent per conubia nostra, per inceptos himenaeos. Vestibulum quis magna at risus dictum tempor eu vitae velit.
%\end{block}

%\begin{block}{Block 3}
%Suspendisse tincidunt sagittis gravida. Curabitur condimentum, enim sed venenatis rutrum, ipsum neque consectetur orci, sed blandit justo nisi ac lacus.
%\end{block}
%\end{frame}

%------------------------------------------------

%\begin{frame}
%\frametitle{Multiple Columns}
%\begin{columns}[c] % The "c" option specifies centered vertical alignment while the "t" option is used for top vertical alignment

%\column{.45\textwidth} % Left column and width
%\textbf{Heading}
%\begin{enumerate}
%\item Statement
%\item Explanation
%\item Example
%\end{enumerate}

%\column{.5\textwidth} % Right column and width
%Lorem ipsum dolor sit amet, consectetur adipiscing elit. Integer lectus nisl, ultricies in feugiat rutrum, porttitor sit amet augue. Aliquam ut tortor mauris. Sed volutpat ante purus, quis accumsan dolor.

%\end{columns}
%\end{frame}

%------------------------------------------------
%\section{Second Section}
%------------------------------------------------

%\begin{frame}
%\frametitle{Table}
%\begin{table}
%\begin{tabular}{l l l}
%\toprule
%\textbf{Treatments} & \textbf{Response 1} & \textbf{Response 2}\\
%\midrule
%Treatment 1 & 0.0003262 & 0.562 \\
%Treatment 2 & 0.0015681 & 0.910 \\
%Treatment 3 & 0.0009271 & 0.296 \\
%\bottomrule
%\end{tabular}
%\caption{Table caption}
%\end{table}
%\end{frame}

%------------------------------------------------

%\begin{frame}
%\frametitle{Theorem}
%\begin{theorem}[Mass--energy equivalence]
%$E = mc^2$
%\end{theorem}
%\end{frame}

%------------------------------------------------

%\begin{frame}[fragile] % Need to use the fragile option when verbatim is used in the slide
%\frametitle{Verbatim}
%\begin{example}[Theorem Slide Code]
%\begin{verbatim}
%\begin{frame}
%\frametitle{Theorem}
%\begin{theorem}[Mass--energy equivalence]
%$E = mc^2$
%\end{theorem}
%\end{frame}\end{verbatim}
%\end{example}
%\end{frame}

%------------------------------------------------

%\begin{frame}
%\frametitle{Figure}
%Uncomment the code on this slide to include your own image from the same directory as the template .TeX file.
%\begin{figure}
%\includegraphics[width=0.8\linewidth]{test}
%\end{figure}
%\end{frame}

%------------------------------------------------

%\begin{frame}[fragile] % Need to use the fragile option when verbatim is used in the slide
%\frametitle{Citation}
%An example of the \verb|\cite| command to cite within the presentation:\\~

%This statement requires citation \cite{p1}.
%\end{frame}

%------------------------------------------------

%\begin{frame}
%\frametitle{References}
%\footnotesize{
%\begin{thebibliography}{99} % Beamer does not support BibTeX so references must be inserted manually as below
%\bibitem[Smith, 2012]{p1} John Smith (2012)
%\newblock Title of the publication
%\newblock \emph{Journal Name} 12(3), 45 -- 678.
%\end{thebibliography}
%}
%\end{frame}

%------------------------------------------------

\begin{frame}
\Huge{\centerline{The End}}
\end{frame}

%----------------------------------------------------------------------------------------

\end{document} 